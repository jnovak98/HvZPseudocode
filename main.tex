\documentclass{article}
\usepackage[utf8]{inputenc}
\usepackage{tikz}
\usepackage{ wasysym }
\usepackage{listings}
\usepackage[margin=1in]{geometry}

\title{Humans vs Zombies Pseudocode}
\author{Jeremy Novak (jrn50) }
\date{October 2018}

\usepackage{natbib}
\usepackage{graphicx}

\begin{document}

\maketitle
\section{Data Structure Design}
To implement the humans vs zombies code, I decided to store Zombie objects in a red-black tree, which is a variety of binary tree. The benefits of doing this is that it natively supports almost all required functions in $O(logN)$ time (just not bomb). For the pseudo code, I will not be covering the innerworking of a red-black binary tree since this is well documented elsewhere. I will assume that it has the following functions: $get$, $put$, $remove$, $first$, $last$, $ceiling$, $floor$, and $subtree$. These functions are easy to implement in Binary trees and as such can be found in almost every implementation of red-black trees, such as Java's $TreeMap$.

\section{Compare $(a,b)$}
For any binary tree, we need to have some way to compare keys. In this case, all functions that look for a zombie other than by searching for a specific coordinate, involve the x-axis. Thus, it makes since to have the red-black tree primarily sort by x-axis, and secondarily by the y-axis.
\begin{enumerate}
    \item \textbf{if} $A_x = B_x$ \textbf{then} \textbf{return} which has the higher y value
    \item \textbf{else return} whichever has the higher x value
\end{enumerate}

\section{Zombie$(x,y)$}
Zombie$(x,y)$  returns  the  zombie  at  location  $(x, y)$  on  the  wall or null if there is no such zombie.
\begin{enumerate}
    \item Call $get$ on red-black tree for a key of  $(x,y)$ and return value. 
    \item \textbf{if} no value is found \textbf{then} \textbf{return} $null$
\end{enumerate}
We know this algorithm works because searching is a native function of every red-black tree ADT. Runtime is $O(log(n))$.

\section{Insert$(z,x,y)$}
Insert$(z,x,y)$ inserts zombie $z$ at location $(x,y)$ on the wall. It returns an error if a zombie is already present at $(x,y)$ or if $z$ is null.
\begin{enumerate}
    \item \textbf{if} $Zombie(x,y)$ doesn't equal null, \textbf{return} an error
    \item \textbf{else} Call $put$ on red-black tree for a key of  $(x,y)$ and a value of $z$
\end{enumerate}
We know insert works because it is a native feature of every red-black tree ADT. Runtime is $O(log(n))$.

\section{Delete$(x,y)$}
Delete$(x,y)$  deletes  and  returns  the  zombie  at  $(x,y)$.  It  returns an error if there is no zombie at that location. 
\begin{enumerate}
    \item$ Z \leftarrow Zombie(x,y)$ 
    \item \textbf{if} $z=null$, \textbf{return} an error
    \item \textbf{else} 
    \begin{enumerate}
        \item Call $remove$ on red-black tree for a key of $(x,y)$
        \item \textbf{return} Z
    \end{enumerate}
\end{enumerate}
We know delete works for the same reason insert works, as its code functions in the same way and $remove$ is a native feature of red-black tree ADTs. Runtime is $O(log(n))$.

\section{Javelin$(x_P)$}
Javelin$(x_P)$ returns the $(x,y)$ coordinates of the zombie that would be killed by the javelin, or null if there is no zombie.The argument represents the player’s position. The javelin  kills  the  zombie  whose  horizontal  position  is  closest to that of the player, irrespective of height. 

\begin{enumerate}
    \item  $left \leftarrow $ get the key of a $floor$ function on value $(x_P)$
    \item  $right \leftarrow $ get the key of a $ceiling$ function on value $(x_P)$
    \item \textbf{if} both  $left$ and $right$ are null \textbf{then return} $null$. 
    \item \textbf{if} one out of $left$ and $right$ is null \textbf{then return} the other. 
    \item \textbf{return} the closer of $left$ and $right$
\end{enumerate}
We know Javelin works because the closest key will either be greater than, less than, or equal. Ceiling and floor functions find the closest greater than or equal key and the closest less than or equal to key. Thus, the left and right keys represent the two possible closest keys, and then we just manually check the closer of the two. Runtime is $O(log(n))$.

\section{Arrow$(direction)$}
Arrow$(direction)$ returns  the  $(x,y)$ coordinate of the zombie that would be killed by the arrow. The direction is either left or right. It returns null if there is no zombie. The arrow  kills  the  zombie  that  is  furthest  to  the  left  (or  the  right, at the player’s choosing), irrespective of his height.

\begin{enumerate}
    \item \textbf{if} direction = left \textbf{then return} furthest left key in tree
    \item \textbf{else return} furthest right key
\end{enumerate}
Arrow works because finding the least and greatest keys in a tree is natively supported by red-black trees, as it just constitutes repeatedly taking the left or right node until one reaches the bottom. As a result, runtime is $O(log(n))$.

\section{Bomb$(x_P,r)$}
Bomb$(x_P, r)$ returns the $(x, y)$ coordinate of the zombie that would be killed by a bomb of range $r$ launched by a player at coordinate $x_P$. It returns $null$ if there is no zombie in range. The bomb  kills the zombie that is the highest on the  wall among all of the zombies within a given horizontal distance $r$ from the player’s position. In other words, if the player horizontal position is $x_P$, the bomb kills the highest zombie whose horizontal coordinate is between $x_P-r$ and $x_P+r$. 

\begin{enumerate}
    \item call subtree function to get all keys greater than $x_P - r$ and less than $x_P + r$ in the tree.
    \item Convert subtree to a set $S$ than can be iterated through
    \item $H \leftarrow null$
    \item \textbf{for} each key $K$ in $S$
    \begin{enumerate}
        \item \textbf{if} $K_y > H_y$ \textbf{then} $H \leftarrow K$ 
    \end{enumerate}
    \item \textbf{return} $H$ 
\end{enumerate}
We know bomb works because it simply compares every key in the range of the bomb and selects the one with the highest y-value. Runtime on this function is the slowest, but is bounded by a worst case runtime of $O(n)$.

\end{document}
